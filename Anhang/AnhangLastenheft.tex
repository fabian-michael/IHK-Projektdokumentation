\subsection{Lastenheft (Auszug)}
\label{app:Lastenheft}
Es folgt ein Auszug aus dem Lastenheft mit Fokus auf die Anforderungen:

Die Anwendung muss folgende Anforderungen erfüllen: 
\begin{itemize}[itemsep=0em,partopsep=0em,parsep=0em,topsep=0em]
	\item Kompabilität
		\begin{itemize}
			\item Die Extension muss sowohl mit TYPO3 8.7 als auch mit TYPO3 9.5 oder höher kompatibel sein.
			\item Es dürfen keine TYPO3 Features/APIs verwendet werden, welche ab TYPO3 9.5 deprecated sind.
		\end{itemize}
	\item Allgemein
		\begin{itemize}
			\item Die Konfiguration und Validierung soll möglichst modular und einfach zu erweitern sein. Sowohl im Front- als auch im Backend.
			\item Die Extension soll ein reines Entwicklerwerkzeug sein. Die Umsetzung einer GUI für den Endnutzer soll individuell in den jeweiligen Projekten geschehen.
			\item Für das Frontend soll lediglich die Core-Funktionalität entwickelt und eine API zur Verfügung gestellt werden.
			\item Das Hochladen der Dateien soll mittels Ajax geschehen.
			\item Zur Integritätssicherung soll eine Checksumme berechnet werden
			\item Es soll ein Viewhelper zur Verfügung gestellt werden, der die nötigen HTML-Elemente erzeugt.
		\end{itemize}
	\item Backend und Konfiguration
		\begin{itemize}
			\item Die Konfiguration soll global über eine Registry geschehen.
			\item Sie soll möglichst modular sein.
			\item Es sollen mehrere Konfigurationen vorgenommen werden können. Jede Konfiguration soll einen eindeutigen Key erhalten.
			\item Folgende Einstellmöglichkeiten sollen standardmäßig existieren:
				\begin{itemize}
					\item Zielverzeichnis
					\item Maximale Dateigröße (einzeln)
					\item Maximale Gesamtgröße
					\item Mindestanzahl an Dateien
					\item Maximalanzahl an Dateien
					\item Erlaubte MIME-Types
					\item Erlaubte Dateiendungen
				\end{itemize}
			\item Darüber hinaus sollen zur Erweiterung eigene Settings registriert werden können. 
		\end{itemize}
\end{itemize}

