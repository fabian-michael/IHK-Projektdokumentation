% !TEX root = ../Projektdokumentation.tex
\section{Analysephase} 
\label{sec:Analysephase}


\subsection{Ist-Analyse} 
\label{sec:IstAnalyse}

Im Rahmen mehrerer Kundenprojekte wurden bereits verschiedene Tools, sowohl Open Source als auch kommerziell, für das Hochladen von Dateien evaluiert und eingesetzt. Sowohl Backend- als auch Frontendseitig hat sich bisher allerdings keine zufriedenstellende Lösung ergeben. Gründe gehen von fehlenden Features bis unzureichenden Dokumentationen, unregelmäßige Wartungen und schlechte Anpassbarkeit. 

Backendseitig ist die Integration in das TYPO3 Ökosystem aufwändig oder es muss für jedes Projekt neu implementiert werden, was zusätzliche Arbeit und kosten bedeutet. Hinzu kommen Security Issues und mangelnde Umsetzung der PSR-Standards.  

Frontendseitig haben sich die bisher getesteten Tools als sehr konfigurationslastig und schlecht anpassbar herausgestellt. Zudem erfolgt die Konfiguration von Backend und Frontend getrennt, was zu Inkonsistenzen führen kann. 


\subsection{Wirtschaftlichkeitsanalyse}
\label{sec:Wirtschaftlichkeitsanalyse}

\subsubsection{\gqq{Make or Buy}-Entscheidung}
\label{sec:MakeOrBuyEntscheidung}
Da bisher keine zufriedenstellende Lösung gefunden wurde, weder Open Source, noch kommerziell und wir als Softwaredienstleister die Möglichkeiten haben das Projekt selbst zu warten und weiter zu entwickeln, wurde auf Eigenentwicklung gesetzt. Damit ist das Projekt außerdem unabhängig von externen Entwicklern. Zudem ermöglicht dies die freie Verwendung in den kommerziellen Kundenprojekten und führt zu einem firmenweiten Standard, was die Wartung der Kundenprojekte, in denen das Tool zum Einsatz kommt deutlich vereinfacht.

\subsubsection{Projektkosten}
\label{sec:Projektkosten}
Die Kosten für die Durchführung des Projekts setzen sich sowohl aus Personal-, als auch aus Ressourcenkosten zusammen.

Die Durchführungszeit des Projekts beträgt 70 Stunden und der Stundenlohn für einen Auszubildenden beträgt etwa \eur{7,56}. Für andere Mitarbeiter wird pauschal ein Stundenlohn von \eur{25} und für die Nutzung von Ressourcen\footnote{Räumlichkeiten, Arbeitsplatzrechner etc.}
ein Stundensatz von \eur{15} angenommen. 
Die Aufstellung der Kosten befindet sich in Tabelle~\ref{tab:Kostenaufstellung} und sie betragen insgesamt \eur{2219,20}.
\tabelle{Kostenaufstellung}{tab:Kostenaufstellung}{Kostenaufstellung.tex}


\subsubsection{Amortisationsdauer}
\label{sec:Amortisationsdauer}

Bei einer Zeiteinsparung von schätzungsweise 10 Stunden pro Dateiupload und 6 Dateiuploads im Jahr ergibt sich eine gesamte Zeiteinsparung von 
\begin{eqnarray}
\mbox{ 6/Jahr} \cdot 10 \mbox{ h} = 60 \mbox{ h/Jahr} 
\end{eqnarray}

Dadurch ergibt sich eine jährliche Einsparung von 
\begin{eqnarray}
60 \mbox{h/Jahr} \cdot \eur{(25 + 15)}{\mbox{/h}} = \eur{2400}\mbox{/Jahr}
\end{eqnarray}

Die Amortisationszeit beträgt also $\frac{\eur{2219,20}}{\eur{2400}\mbox{/Jahr}} \approx 0,92 \mbox{ Jahre} \approx 11 \mbox{ Monate}$.


\subsection{Nicht-Monetäre Vorteile}
Die Nicht-Monetären Vorteile sind wie bereits aufgeführt die einfache Handhabung des Tools für Entwickler und das Schaffen eines firmenweiten Standards, was die Wartung und Einführung deutlich vereinfacht.


\subsection{Anwendungsfälle}
\label{sec:Anwendungsfaelle}

Zur Veranschaulichung der Anwendungsfälle wurde ein Anwendungsfalldiagramm erstellt, das im \Anhang{app:UseCase} eingesehen werden kann. Dabei werden Anwendungsfälle aus Sicht eines Endnutzers und aus Sicht eines Entwicklers dargestellt. Der Anwendungsfall des Hochladens wurde außerdem exemplarisch in einer EPK modelliert. Diese ist im \Anhang{app:EPK} zu finden. Sie ist in drei Spalten gegliedert. Die erste Spalte beschreibt die Prozesse und Ereignisse vor, die mittlere Spalte während und die dritte Spalte nach dem Hochladen.


\subsection{Qualitätsanforderungen}
\label{sec:Qualitaetsanforderungen}

Da das Tool in erster Linie ein Entwicklerwerkzeug ist, sind die wichtigsten Qualitätsanforderungen an das Projekt:
\begin{itemize}
	\item Zuverlässigkeit – Entwickler sollten sich auf die Core-Funktionalität und die API des Fileuploaders verlassen können.
	\item Benutzbarkeit – Es sollte eine steile Lernkurve und simple API besitzen.
	\item Übertragbarkeit – Es sollte sich problemlos in Projekte integrieren lassen.
	\item Änderbarkeit – Durch Modularität soll gewährleistet sein, dass Entwickler einfach projektspezifische Erweiterungen vornehmen können.
\end{itemize}


\subsection{Lastenheft/Fachkonzept}
\label{sec:Lastenheft}

Nach der Analysephase und dem Fachgespräch mit dem Fachbereich wurde ein Lastenheft erstellt, das alle Anforderungen an das Tool umfasst. Im \Anhang{app:Lastenheft} kann ein Auszug aus diesem eingesehen werden.