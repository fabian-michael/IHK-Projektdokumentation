% !TEX root = ../Projektdokumentation.tex
\section{Implementierungsphase} 
\label{sec:Implementierungsphase}

\subsection{Implementierung der Geschäftslogik}
\label{sec:ImplementierungGeschaeftslogik}

\subsubsection{Die IDE}
Als Entwicklungsumgebung wurde PHPStorm verwendet. PHPStorm ist eine kommerzielle IDE entwickelt von Jetbrains. Diese IDE bietet einen großen Funktionsumfang und hervorragende Unterstützung beim Entwickeln von Webprojekten. Wie der Name bereits andeuten lässt, ist PHPStorm gut geeignet für die Entwicklung mit PHP. Da PHPStorm auf WebStorm basiert, liefert es außerdem Tools und Unterstützung für Front-End-Technologien, wie JavaScript und TypeScript.

\subsubsection{Implementierung}

\paragraph{Backend} Im ersten Schritt wurde mit der Implementierung der Backend-Logik begonnen. Zunächst wurden die Konfigurations- und Processor-API implementiert, da dies den Grundstein für das Projekt legt. 

Anschließend konnte mit dem View Helper fortgefahren werden. Da ein View Helper normalerweise nur einen HTML-Tag rendert, musste etwas getrickst werden, damit dieser die zwei erforderlichen HTML-Tags rendert. Der View Helper konnte dann in der Testumgebung in ein Template eingebunden werden. 

Im nächsten Schritt konnte damit begonnen werden den Endpoint einzurichten, an den die HTTP-Requests gesendet werden sollen. Dies geschieht, indem im TypoScript einen sogenannter PAGE-Type mit einer eindeutigen Nummer (typeNum) angelegt wird. Dieser PAGE-Type wurde so konfiguriert, dass die handleRequest-Methode des Ajax-Handlers aufgerufen werden soll, sobald ein HTTP-Request an diesen gesendet wird. Dies geschieht mittels des Query-Parameters ?type=[typeNum des PAGE-Types] in der URL. Für das Aufrufen von PHP-Funktionen gibt es den USER- bzw. USER\_INT-Type. In dem Fall wurde USER\_INT-Type verwendet, da dieser im gegensatz zu USER nicht gecached wird. In \Abbildung{typoscript}: TypoScript Konfiguration ist die Konfiguration ersichtlich. 

\begin{figure}[htb]
\centering
\caption{TypoScript Konfiguration}
\includegraphicsKeepAspectRatio{typoscript.png}{0.9}
\label{fig:typoscript}
\end{figure}

Nachdem dies erledigt war, konnten der Ajax-Handler, der Service und abschließend die ``Common'' Processor implementiert werden. Damit war das Backend grundsätzlich fertig.

Schwierigkeiten ergaben sich bei der Implementierung der Konfigurations-API, da diese sehr modular aufgebaut ist, es allerdings Stellen gab, an denen es Abhängigkeiten zu bestimmten Einstellungen gab. Dabei wurde einiges an Zeit vergeuded, durch häufiges Änderung und Refaktorisieren, bis sich am Ende eine zufriedenstellende Lösung ergeben hat.

\paragraph{Frontend} Nun konnte mit der Implementierung des Frontends begonnen werden. Das Frontend basiert auf Modulen. Ein Modul ist grundsätzlich eine JavaScript- bzw TypeScript-Datei, die Daten bzw. Funktionen aus anderen Modulen importiert und/oder Daten bzw. Funktionen exportiert, sodass andere Module diese importieren können. Die Implementierung erfolgte Angefangen beim Entry-Point-Modul über die Processors Registry und API gefolgt vom Herzstück, der Fileuploader- und  Store-Klasse. Abschließend wurden die ``Common'' Processors implementiert, die für das Frontend relevant sind.

\subsubsection{Fazit zur Implementierungsphase}
\label{sec:FazitImplementierung}

Abschließend hat die Implementierungsphase länger gedauert als Ursprünglich geplant. Unter anderem durch Schwierigkeiten bei der Konfigurations-API und einer generell unterschätzten Zeitplanung.