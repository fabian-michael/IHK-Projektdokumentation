% !TEX root = ../Projektdokumentation.tex
\section{Einleitung}
\label{sec:Einleitung}
Diese Projektdokumentation schildert den Ablauf des Abschlussprojektes, das vom Autor im Rahmen der Ausbildung zum Fachinformatiker für Anwendungsentwicklung durchgeführt wurde. Der Ausbildungsbetrieb des Autors ist die form4 GmbH \& Co. KG aus Berlin.

\subsection{Projektumfeld} 
\label{sec:Projektumfeld}
1996 als Internetagentur gegründet hat die form4 GmbH \& Co. KG sich zu einem Softwareunternehmen entwickelt. Die Schwerpunkte liegen auf der Herstellung von Websites und Portalen sowie der Realisierung individueller, webbasierter Softwarelösungen für mittelständische Unternehmen und Organisationen. Das Unternehmen setzt sich aus zwei Abteilungen - im Folgenden als Units bezeichnet - zusammen. Die Web-Unit und die Enterprise-Unit. Der Auftraggeber dieses Projekts ist der Fachbereich TYPO3 bestehend aus den Seniorentwicklern der Web-Unit.

\paragraph{Die Web-Unit}
Schwerpunkt der Webunit ist die technische Umsetzung von Websites und Portalen im öffentlich zugänglichen Bereich. Hierunter fallen auch Community- und eCommerce-Lösungen.

Kerntechnologien sind hierbei PHP, HTML, CSS und JavaScript.

Die meisten von der Webunit realisierten Projekte basieren auf leistungsfähigen Content Management Systemen wie TYPO3 oder Drupal.

\paragraph{Die Enterprise-Unit}
Die Enterprise-Unit beschäftigt sich in erster Linie mit javabasierten Softwaresystemen. Das beinhaltet Intranet- und Unternehmensportale wie auch klassische individuelle datenverwaltende Software und Workflowsysteme.

\subsection{Projektziel} 
\label{sec:Projektziel}
Ziel des Projektes ist die Entwicklung einer TYPO3-Extension und dem dazugehörigen Frontend (ohne GUI). Diese Extension soll in Kundenprojekten zum Produktiveinsatz kommen und bestehende Implementierungen ablösen. Dadurch sollen Entwickler zukünftig entlastet werden und eine einfache und einheitliche Lösung zur Verwaltung von asynchronen Datei-Uploads geschaffen werden. 


\subsection{Projektbegründung} 
\label{sec:Projektbegruendung}
Die Motivation dieses Projektes ist es, durch ein einheitliches, einfach zu integrierendes, konfigurierendes und erweiterndes Tool einen firmenweiten Standard zu schaffen. Durch die zentrale und modular gestaltete Konfigurationsschnittstelle soll der Aufwand und die Fehleranfälligkeit bei der Implementierung eines Dateiuploads deutlich reduziert werden, sodass sich die Entwickler auf das wesentliche konzentrieren können.


\subsection{Projektschnittstellen} 
\label{sec:Projektschnittstellen}
Da es sich um eine TYPO3 Extension handelt, verwendet das Projekt dementsprechend die APIs, die von TYPO3 zur Verfügung gestellt werden. TYPO3 ist ein Open Source Enterprise \ac{CMS} für die Umsetzung von kleinen bis hin zu sehr großen multinationalen Websites. Nach außen stellt das Projekt sowohl im Backend als auch im Frontend für die Entwickler eine API bereit, mit der es in ein Kundenprojekt integriert werden kann. Unter anderem im Backend die Konfigurationsschnittstelle, mit der zentral die Konfigurationen verwaltet werden können und eine ``Processors''-API zur Verarbeitung von Dateien. Im Frontend gibt es ebenfalls eine Processors-API und desweiteren Schnittstellen zum Realisieren einer GUI und zur Fehlerbehandlung.


\subsection{Projektabgrenzung} 
\label{sec:Projektabgrenzung}
Nicht teil dieses Projektes ist die Umsetzung einer GUI. Es sollen lediglich die notwendigen HTML-Elemente erzeugt werden. Dies sind die Formular-Felder, mit denen der Benutzer Dateien auswählen kann und die Daten beim Absenden des Formulars an das Backend gesendet werden. Mehr dazu in Abschnitt \ref{sec:Entwurfsphase}: Entwurfsphase. Das JavaScript beschränkt sich nur auf die Verarbeitung und Validierung der Daten, dem State Management und stellt eine API bereit, mit deren Hilfe dann jeweils eine ansprechende und Projektspezifische GUI umgesetzt werden kann.
